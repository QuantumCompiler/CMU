%---------------------------------------------------------------------------
%	Packages
%---------------------------------------------------------------------------
\documentclass{article}
\usepackage[bottom]{footmisc}
\usepackage[affil-it]{authblk}
\usepackage{amsmath}
\usepackage{setspace}
\usepackage{url}
\usepackage{amsthm}
\usepackage{tikz}
\usetikzlibrary{shapes.geometric, arrows}
\usetikzlibrary{decorations.pathreplacing}
\usetikzlibrary{calc}
\usepackage{pgfplots}
\usepgfplotslibrary{units}
\usepackage{indentfirst}
\usepackage{gensymb}
\pgfplotsset{compat=1.10}
\usepackage{amsmath}
\usepackage{tikz}
\usetikzlibrary{shapes.geometric, arrows}
\usetikzlibrary{decorations.pathreplacing}
\usetikzlibrary{decorations.pathmorphing}
\usetikzlibrary{decorations.markings}
\usepackage{pgfplots}
\usepackage{pgf}
\usepackage{fancyhdr}
\usepgfplotslibrary{units}
\pgfplotsset{compat=1.10}
\usepgfplotslibrary{units}
\usepackage{tkz-euclide}
\usepackage{nopageno}
\usepackage{xcolor}
\usepackage{graphicx}
\usetikzlibrary{arrows.meta}
\tikzset{>=Stealth}
\tikzset{snake it/.style={decorate, decoration=snake}}
%---------------------------------------------------------------------------
%	Header and footer
%---------------------------------------------------------------------------
\pagestyle{fancy}
\lhead{\small{Coupled Oscillations of Pendulums}}
\chead{\small{T.J Larrechea}}
\rhead{\small{Colorado Mesa University}}
%---------------------------------------------------------------------------
%	Title and Author
%---------------------------------------------------------------------------
\title{\textbf{Quantum Parameter Estimation}}
\author{Taylor Larrechea\footnote{Electronic Address: \texttt{tjlarrechea@gmail.com}} \\
    Colorado Mesa University \\
    Physical and Environmental Sciences \\
    1100 North Ave. Wubben Science \\
    Grand Junction, CO 81501}
\date{\today}
%---------------------------------------------------------------------------
%	Document
%---------------------------------------------------------------------------
\begin{document}
\maketitle
Quantum parameter estimation is the method of which quantum mechanical systems are used as devices to measure physical parameters. The parameters that are being estimated arise in various physical processes. Estimation accuracy is quantified by variance, which is bound by Quantum Fisher Information which can be calculated from the state of the system. The Quantum Fisher Information depends on whether single or multiple particles are used as probes and whether their states are correlated or independent. Sometimes certain multiple particle states yield greater accuracy comparable to classical particle states. Noisy states can mean the initial state is not perfectly known. Estimation has been partially assessed when available particle states are noisy. When specific channels incorporate additional noise the Quantum Fisher Information typically decreases rather than when the channels leave the particles undisturbed. When multiple particles are introduced that undergo phase flips, it has been shown to be advantageous for certain strengths of phase flips whereas depolarizing channels seldomly yield a better estimation than that of other methods.
\end{document}