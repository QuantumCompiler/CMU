%---------------------------------------------------------------------------
%	Packages
%---------------------------------------------------------------------------
\documentclass[twocolumn]{article}
\usepackage[bottom]{footmisc}
\usepackage[affil-it]{authblk}
\usepackage{amsmath}
\usepackage{setspace}
\usepackage{url}
\usepackage{amsthm}
\usepackage{tikz}
\usetikzlibrary{shapes.geometric, arrows}
\usetikzlibrary{decorations.pathreplacing}
\usetikzlibrary{calc}
\usepackage{pgfplots}
\usepgfplotslibrary{units}
\usepackage{indentfirst}
\usepackage{gensymb}
\pgfplotsset{compat=1.10}
\usepackage{amsmath}
\usepackage{tikz}
\usetikzlibrary{shapes.geometric, arrows}
\usetikzlibrary{decorations.pathreplacing}
\usetikzlibrary{decorations.pathmorphing}
\usetikzlibrary{decorations.markings}
\usepackage{pgfplots}
\usepackage{pgf}
\usepackage{fancyhdr}
\usepgfplotslibrary{units}
\pgfplotsset{compat=1.10}
\usepgfplotslibrary{units}
\usepackage{tkz-euclide}
\usetkzobj{all}
\usepackage{xcolor}
\usepackage{graphicx}
\usetikzlibrary{arrows.meta}
\tikzset{>=Stealth}
\tikzset{snake it/.style={decorate, decoration=snake}}
%---------------------------------------------------------------------------
%	Header and footer
%---------------------------------------------------------------------------
\pagestyle{fancy}
\lhead{\small{Coupled Oscillations of Pendulums}}
\chead{\small{T.J Larrechea}}
\rhead{\small{Colorado Mesa University}}
%---------------------------------------------------------------------------
%	Title and Author
%---------------------------------------------------------------------------
\begin{document}
\title{\textbf{PHYS 482 Literature Review}}
\author{Taylor Larrechea\footnote{Electronic Address: \texttt{tjlarrechea@mavs.coloradomesa.edu.}} \\
    Colorado Mesa University \\
    Department of Physical and Environmental Sciences \\
    1100 North Avenue \\
    Grand Junction, CO 81501-3122}
\date{\today}
\maketitle
%---------------------------------------------------------------------------
%	Abstract
%---------------------------------------------------------------------------
\begin{abstract}
The purpose of this paper is to review an article that was examined for PHYS 482 Senior Research at Colorado Mesa University. The article that is being reviewed is, ''Qubit-channel Metrology With Very Noisy Initial States" by Professor David Collins.
\end{abstract}
%---------------------------------------------------------------------------
%	Article Review
%---------------------------------------------------------------------------
\section*{Article Review}
The article that is being reviewed for this paper focuses on Quantum parameter estimation. Precisely, Qubit-channel metrology that is intially in very noisy states. The term, Quantum parameter estimation or metrology for short, refers to using quantum systems as physical measuring devices. The parameter of question must be inferred from measurements on the physical system. One question that this article addresses is if systems such as entangled states enhance the estimation accuracy compared to classical strategies using uncorrelated states. Some states such as Nuclear Magnetic Resonance, or NMR for short, do not have pure states. Instead these NMR solution-states bring to question whether correlating mixed or noisy states have an advantage in performing these measurements. This same article states that there is no general result for all single-qubit channels and for all initial states. The biggest question that this article addresses is for any single qubit-channel, does there exist a parameter estimation protocol that uses correlated states and also provides accuracy enhancements in parameter estimation.

This article ties into the research that I am doing because it talks about first and foremost quantum parameter estimation. My research mainly focuses on how to improve estimation via metrology. My research is in its infancy so I am not able to explain to the fullest extent what I am trying to find as to how I am trying to find it.

This article concludes that for nonunital channels with a parameter-dependent shift there is no improvement in accuracy using specific parameters. A lot of what determines the success of measurements is dependent upon the initial noise of the state. There have been plenty studies of parameter estimation in presence of noisy processes. Although certain parts of each study might be different, they all typically depend upon the initial noise of the state in how well the estimation process is performed.
%---------------------------------------------------------------------------
%	The Bibliography
%---------------------------------------------------------------------------
\newpage
\begin{thebibliography}{99}
Collins, David. “Qubit-Channel Metrology with Very Noisy Initial States.” Physical Review A, vol. 99, no. 1, 29 Jan. 2019, doi:10.1103/physreva.99.012123.
\end{thebibliography}
%---------------------------------------------------------------------------
%	End Document
%---------------------------------------------------------------------------
\end{document}
%---------------------------------------------------------------------------
%	Comment Headers
%---------------------------------------------------------------------------

%---------------------------------------------------------------------------
%	
%---------------------------------------------------------------------------